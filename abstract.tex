\section{Abstract}

Here we describe the updated MolProbity rotamer-library distributions derived from an order-of-magnitude larger and more stringently quality-filtered dataset of about 8000 (vs. 500) protein chains, and we explain the resulting changes and improvements to model validation as seen by users. To include only sidechains with satisfactory justification for their given conformation, we added residue-specific filters for electron-density value and model-to-density fit.  The combined new protocol retains a million residues of data, while cleaning up noise in the multi-dimensional $\chi$ distributions.  It enables clean characterization of conformational clusters nearly 1000-fold less frequent than the most common ones.  We describe examples of local interactions that favor these rare conformations, including the role of authentic covalent bond-angle deviations in enabling presumably strained sidechain conformations. Further, along with \textit{favored} and \textit{outlier}, an \textit{allowed} category (0.3\% to 2.0\% occurrence in reference data) has been added, analogous to Ramachandran validation categories. The new rotamer distributions are used for current rotamer validation in MolProbity and \textit{PHENIX}, and for rotamer choice in \textit{PHENIX} model-building and refinement. The multi-dimensional $\chi$ distributions and Top8000 reference dataset are available on GitHub. These rotamers are termed "ultimate" because data sampling and quality are now fully adequate for this task, and also because we believe the future of conformational validation should integrate sidechain and backbone.